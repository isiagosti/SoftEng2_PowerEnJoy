\section{INTRODUCTION}

\subsection{Revision History}
\begin{table}[h]
	\centering
	\begin{tabular}{| m{1.5cm} | m{1.6cm} | m{5cm} | m{3.9cm} |}
		\hline
		\textbf{Version} & \textbf{Date} & \textbf{Author(s)} & \textbf{Summary}\\
		\hline
		1.0 & 22/01/17 & Isabella Agosti, Carolina Cattivelli & Initial release\\
		\hline
		1.1 & 02/02/17 & Isabella Agosti, Carolina Cattivelli & Notification Manager added in Report Issue (EI)\\
		\hline
	\end{tabular}
\end{table}

\subsection{Purpose and Scope}
The purpose of this document is to define the plan for the PowerEnJoy project, identifying the tasks to be completed, the risks that might occur during its development and the costs of its development. This information can be subsequently used as a guidance to define the required budget, the resources allocation and the schedule of the activities.

\subsection{Document organization}
The document is organized as follows:
\begin{itemize}
	\item Section 1, \textit{Introduction}, gives an overview of this document describing its contents, scope etc.
	\item Section 2, \textit{Project size, cost and effort estimation}, presents an estimate of the expected size of PowerEnJoy in terms of lines of code and of the cost/effort required to actually develop it, based on Function Points and COCOMO approaches.
	\item Section 3, \textit{Schedule}, presents a possible schedule for the project, covering all activities from the requirements identification to the implementation and testing.
	\item Section 4, \textit{Resource Allocation}, describes how the tasks will be assigned to each member of the development team.
	\item Section 5, \textit{Risk Management}, presents the possible risks that PowerEnJoy could face during the development of the project.
	\item Section 6, \textit{Effort Spent}, includes information on the number of hours each group member has worked towards the fulfillment of this deadline. 
\end{itemize}

\newpage
\subsection{Definitions, Acronyms, Abbreviations}
\subsubsection{Definitions}
\begin{center}
	\begin{tabular} { | m{4cm} | m{9cm} | }
		\hline
		\textbf{Keyword} & \textbf{Definitions}\\
		\hline
		User & A person that interacts with the PowerEnJoy mobile or web
		application to register to the system.\\
		\hline
		Registered user & A person who already registered to the system, that interacts with the PowerEnJoy mobile or web application in various ways.\\
		\hline
		Employee & A member of the PowerEnJoy staff.\\
		\hline
		Car-sharing service & Model of car rental where people rent cars for short periods of time, often by the hour.\\
		\hline
		Electric car & Automobile that is propelled by one or more electric motors, using electrical energy stored in rechargeable batteries.\\
		\hline
		Registration & The act or process of filling out an online form providing credentials and payment information.\\
		\hline
		Log-in & Process by which a user gains access to the system by identifying and authenticating himself/herself. \\
		\hline
		Reservation & Arrangement through which a registered user holds a car for his use at a later time.\\
		\hline
		Safe area & Area whose position is predefined by the management system. Safe areas are the only ones in which a user is allowed to park a car.\\
		\hline
		Special safe area & Special type of safe area where a car can be recharged.\\
		\hline
		Discount percentage & Discount applied on the user’s last ride only in certain circumstances.\\
		\hline
		Low battery & The car’s battery level is considered ”low” when less than 20\%.\\
		\hline
	\end{tabular}
\end{center}
\newpage
\subsubsection{Acronyms and Abbreviations}
\begin{center}
	\begin{tabular} { | m{5cm} | m{8cm} | }
		\hline
		\textbf{Acronym/Abbreviation} & \textbf{Definition}\\
		\hline
		RASD & Requirements Analysis and Specification Document\\
		\hline
		DD & Design Document\\
		\hline
		ITPD & Integration Test Plan Document\\
		\hline
		PP & Project Plan\\
		\hline
		AA & Anno Accademico (Academic Year)\\
		\hline
		DB & Database\\
		\hline
		FP & Function Points\\
		\hline
		ILF & Internal Logic File\\
		\hline
		ELF & External Logic File\\
		\hline
		EI & External Input\\
		\hline
		EO & External Output\\
		\hline
	\end{tabular}
\end{center}

\subsection{Reference Documents}
\begin{itemize}
	\item The project description document: \textit{Specifications document: Assignments AA 2016-2017.pdf}.
	\item The PowerEnJoy Requirements Analysis and Specification Document: \textit{RASD.pdf}.
	\item The PowerEnJoy Design Document: \textit{DD.pdf}.
	\item The Integration Test Plan Document: \textit{ITPD.pdf}.
	\item The Project Planning example: \textit{Project planning example document.pdf}.
	\item The \textit{COCOMO II Model Definition Manual (version 2.1, 1995 – 2000 Center for Software Engineering, USC)}.
\end{itemize}