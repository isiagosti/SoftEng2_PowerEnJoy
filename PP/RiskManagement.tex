\section{RISK MANAGEMENT}
In this paragraph we describe the main risks that the project development may face. Some of them represent technical issues, while others are related to political or financial challenges.

\subsection{Economical Risks}
In this subsection we describe the economical risks PowerEnJoy could face.

\subsection*{Competitors}
It may happen that other companies offer better products at a lower price, causing the exclusion of PowerEnJoy from the market. 

In order to avoid this risk, PowerEnJoy should frequently improve the quality of its products trying to meet its users' needs. 

\subsection*{Lack of fundings}
It may happen that the stakeholders are such dissatisfied with the PowerEnJoy service that they decide to cut its fundings. 

In order to avoid this risk, the PowerEnJoy company should plan monthly meetings with its stakeholders, to let them express their opinion about the service.

\subsection*{Market risks}
It may happen that people are not interested in using our application and prefer traditional methods, causing the company's income to decrease.

A good feasibility study helps avoid this risk. 

\subsection*{Changes in regulation}
Another issue concerns the possible changes in local and State regulators, that could change the PowerEnJoy regulation at any time. 

The only thing we can do in order to avoid this risk is to keep an eye on these laws, which typically take months to be approved, and be ready to move fast before the legislation is actually enacted.
\newpage
\subsection{Project Risks}
In this subsection we describe the project risks PowerEnJoy could face.

\subsection*{Project scheduling}
Even though an initial overall schedule is provided in this document, it may happen that the project requires more time than expected, due to possible issues that may arise during its development.

In order to avoid this risk, some extra time should been allocated at the end of each major activity to consider the possibility of refinements. 

\subsection*{Changes in requirements}
It may happen that the client changes his mind about requirements during the development of the project.

This risk cannot be prevented, but it can be mitigated by writing reusable code.

\subsection*{Lack of experience}
It may happen that our programmers and engineers' knowledge regarding a specific matter or programming technique is overestimated. 

In order to avoid this risk, the company should worry about hiring qualified personnel from the beginning. 

\subsection*{Lack of communication}
It may happen that the personnel has to work remotely, causing misunderstandings of various kind. 

In order to avoid this risk, the company should clearly define the work division among its personnel and have a clear idea regarding the required requirements.
\newpage
\subsection{Technical Risks}
In this subsection we describe the technical risks PowerEnJoy could face.

\subsection*{Loss of data}
It may happen that the company looses some data due to hardware failure, misconfigured software or external attacks. 

In order to avoid this risk, the company should make use of reliable backup techniques distributed over multiple locations, far from the system.

\subsection*{Downtime}
It may happen that the system goes down for any reason, such as excessive load, software bugs, hardware failure or power outages. 

In order to avoid this risk, the company should build multiple and redundant systems and perform testing at all levels.

\subsection*{Integration testing failure}
It may happen that the components do not pass the integration testing phase after their implementation. 

In order to avoid this risk, the company should perform an early integration testing making use of stubs and drivers, and define in details the interfaces between components and subsystems. 