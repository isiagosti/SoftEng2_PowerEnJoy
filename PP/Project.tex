\section{PROJECT SIZE, COST AND EFFORT ESTIMATION}
This section provides some estimations on the expected size, cost and required effort of the PowerEnJoy project.

For the size estimation part we will use the Function Points approach, considering all the main PowerEnJoy functionalities and estimating the correspondent amount of lines of code to be written in Java. 

For the cost and effort estimation we will instead rely on the COCOMO approach, using as initial guidance the amount of lines of code computed with the FP approach.

\subsection{Size estimation: Function Points}
The Function Points approach provides an estimation of a project size taking as inputs the amount of functionalities to be developed and their complexity.

The complexity is evaluated based on the characteristics of the application and described in the following table:


\subsubsection{Internal Logic Files (ILFs)}

\subsubsection{External Logic Files (ELFs)}

\subsubsection{External Inputs (EIs)} 

\subsubsection{External Inquiries (EQs)}

\subsubsection{External Outputs (EOs)} 

\subsubsection{Overall estimation}

\subsection{Cost and effort estimation: COCOMO II}

\subsubsection{Scale Drivers}

\subsubsection{Cost Drivers}

\subsubsection{Effort equation}

\subsubsection{Schedule estimation}
