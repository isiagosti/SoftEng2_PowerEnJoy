\section{INTEGRATION STRATEGY}
\label{integrationstrategy}
\subsection{Entry Criteria}
In order for the integration testing to be possible and to produce meaningful results, there are some criteria that  must be met.

First of all, \textit{RASD} and \textit{DD} must have been fully written. This is required in order to have a complete overview of the interactions between the different components of the system and of the functionalities they offer.

Secondly, all the application's functionalities and components must be able to correctly interact with one another. 
   
\subsection{Elements to be Integrated}
The components that need to be  integrated, referring to our \textit{DD}, are:

\begin{itemize}
	\item \textbf{Client}, that provides a \textbf{User Interface} and a connection between client and server through the \textbf{Connection Manager}. 
	\item \textbf{Employee Manager}, that provides an \textbf{Account Manager} for the login process and a \textbf{Car Manager} to manage the information of the company's cars.
	\item \textbf{User Manager}, that provides a \textbf{Registration Manager}, \textbf{Notification Manager} and \textbf{Email Gateway} for the registration process and an \textbf{Account Generator} for the first login.
	\item \textbf{Registered User Manager}, that provides an \textbf{Account Manager} for the login, view profile, manage personal information, view promotions, view reservations list functionalities, a \textbf{Reservation Manager}, \textbf{Car Manager}, \textbf{Notification Manager} and \textbf{Email Gateway} to cancel reservations and report issues.
	\item \textbf{Ride Manager}, that provides a \textbf{Reservation Manager} to make a reservation, a \textbf{Car Manager} to generate a map with the available cars, a \textbf{Reservation Generator} to create a reservation, a \textbf{Notification Manager} and \textbf{Email Gateway} to send a notification in case a car is abandoned for more than ten minutes.
\end{itemize}
\newpage
\subsection{Integration Testing Strategy}
We will adopt a bottom-up testing strategy, integrating first the subcomponents and then the higher level components. 

Using this approach, we will start integrating together those components that do not depend on others, or that only depend on already developed components.  

Doing so, we can start performing the integration testing as soon as the required components have been developed, in order to maximize parallelism and efficiency.

\subsection{Sequence of Component/Function Integration} 
Since we chose to adopt a bottom-up strategy, we will start from the lower level component and then move upward. 

As a notation, an arrow going from component C1 to component C2 means that C1 is necessary for C2 and so it must have already been implemented.

\subsubsection{Software  Integration  Sequence}
In this section we identify the sequence in which the software components will be integrated within each  subsystem.

\subsubsection*{Database Access System}
The first two components to be integrated are the \textbf{Database} and the \textbf{Database Access}. We start from here because every other component relies on the \textbf{Database Access System} to perform queries.

\begin{figure}[H]
	\centering
	\includegraphics[width=10cm]{DatabaseAccessSystem}
\end{figure}
\newpage
\subsubsection*{User Manager System}
Since the user is the main actor of our application, the second step in the integration process is to appropriately connect the subcomponents implementing the \textbf{User Manager System}. 

In the following diagrams, we show which components are going to be integrated together and in which order, following the bottom-up approach.

First, we proceed by integrating together the \textbf{Account Generator} subcomponent with the \textbf{Database Access System}.
\begin{figure}[H]
	\centering
	\includegraphics[width=10cm]{UserManagerAccountGenerator}
\end{figure}

Then, we proceed by integrating together the \textbf{Registration Manager} subcomponent with the \textbf{Database Access System} and the \textbf{Notification Manager}.
\begin{figure}[H]
	\centering
	\includegraphics[width=14cm]{UserManagerRegistrationManager}
\end{figure}

At this point, the two subcomponents of the \textbf{User Manager System} are ready to be integrated together.
\begin{figure}[H]
	\centering
	\includegraphics[width=14cm]{UserManagerSystem}
\end{figure}

\subsubsection*{Registered User Manager System}
The third step in the integration process is to appropriately connect the subcomponents implementing the \textbf{Registered User Manager System}.

In the following diagrams, we show which components are going to be integrated together and in which order, following the bottom-up approach.

First, we proceed by integrating together the \textbf{Car Manager} subcomponent with the \textbf{Database Access System}.
\begin{figure}[H]
	\centering
	\includegraphics[width=10cm]{RegUserMgrCarManager}
\end{figure}

Then, we proceed by integrating together the \textbf{Reservation Manager} subcomponent with the \textbf{Database Access System}.
\begin{figure}[H]
	\centering
	\includegraphics[width=10cm]{RegUserMgrReservationManager}
\end{figure}

Then, we proceed by integrating together the \textbf{Account Manager} subcomponent with the \textbf{Database Access System}.
\begin{figure}[H]
	\centering
	\includegraphics[width=14cm]{RegUserMgrAccountManager}
\end{figure}
\newpage
At this point, the three subcomponents of the \textbf{Registered User Manager System} are ready to be integrated together.
\begin{figure}[H]
	\centering
	\includegraphics[width=14cm]{RegisteredUserManagerSystem}
\end{figure}
\newpage
\subsubsection*{Employee Manager System}
The fourth step in the integration process is to appropriately connect the subcomponents implementing the \textbf{Employee Manager System}.

In the following diagrams, we show which components are going to be integrated together and in which order, following the bottom-up approach.

First, we proceed by integrating together the \textbf{Car Manager} subcomponent with the \textbf{Database Access System}.
\begin{figure}[H]
	\centering
	\includegraphics[width=10cm]{EmployeeMgrCarManager}
\end{figure}

Then, we proceed by integrating together the \textbf{Account Manager} subcomponent with the \textbf{Database Access System}.
\begin{figure}[H]
	\centering
	\includegraphics[width=10cm]{EmployeeMgrAccountManager}
\end{figure}

At this point, the two subcomponents of the \textbf{Employee Manager System} are ready to be integrated together.
\begin{figure}[H]
	\centering
	\includegraphics[width=10cm]{EmployeeManagerSystem}
\end{figure}
\newpage
\subsubsection*{Ride Manager System}
The last step in the integration process is to appropriately connect the subcomponents implementing the \textbf{Ride Manager System}.

In the following diagrams, we show which components are going to be integrated together and in which order, following the bottom-up approach.

First, we proceed by integrating together the \textbf{Reservation Generator} subcomponent with the \textbf{Database Access System}.
\begin{figure}[H]
	\centering
	\includegraphics[width=9cm]{RideMgrReservationGenerator}
\end{figure}

Then, we proceed by integrating together the \textbf{Car Manager} subcomponent with the \textbf{Database Access System}.
\begin{figure}[H]
	\centering
	\includegraphics[width=9cm]{RideMgrCarManager}
\end{figure}

Then, we proceed by integrating together the \textbf{Reservation Manager} subcomponent with the \textbf{Database Access System} and the \textbf{Notification Manager}.
\begin{figure}[H]
	\centering
	\includegraphics[width=10cm]{RideMgrReservationManager}
\end{figure}

At this point, the three subcomponents of the \textbf{Ride Manager System} are ready to be integrated together.
\begin{figure}[H]
	\centering
	\includegraphics[width=14cm]{RideManagerSystem}
\end{figure}
\newpage
\subsubsection{Subsystem  Integration  Sequence} 
In the following diagram we present the  order  in  which  subsystems  will  be  integrated to create the full PowerEnJoy infrastructure.

\begin{figure}[H]
	\centering
	\includegraphics[width=14cm]{SubsystemIntegrationSequence}
\end{figure}